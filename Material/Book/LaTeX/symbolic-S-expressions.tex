§§ ביטויי~\E|S| כשפה פורמלית

ביטויי~\E|S| הוצגו כבר ב\פנה|פרק:S0| כעצים בינאריים מלאים אשר העלים שלהם נושאים
תגיות. הנה הגדרה מדוית יותר: נשתמש במונח \ע|אלפאבית| לציון קבוצה, סופית או
אינסופית, של סימבולים אשר אין להם משמעות לבד כך שהם שונים זה מזה. בהינתן
אלפאבית~$Γ$ נסמן ב- \[
𝓢=𝓢(Γ)
\]
את \textbf{קבוצת ביטויי ה-\E|S| מעל~$Γ$}. כזכור, ביטוי~\E|S|
יכול להיות אטומי, כלומר כזה שאינו ניתן לפירוק לביטויי~S אחרים. ביטוי אטומי חייב
להיות סימבוך מתוך~$Γ$. ביטוי~\E|S| שאינו אטומי נקרא ביטוי מורכב, ואז הוא בהכרח
זוג סדור של שני ביטויי~\E|S| אחרים, שיכולים להיות אטומיים או מורכבים. הזוג
הסדור נכתב עטוף בזוג סוגריים כאשר שני ביטויי ה-\E|S| שבו מופרדים בסימן הנקודה.

כמה ביטויי~\E|S| מעל האלפאבית (הסופי)~$Γ=❴a,b,c❵$ הם \[
  a,b,(a.b),(c.(b.a)),((a.b).(a.c))∈𝓢❨❴a,b,c❵❩.
\] לעומת זאת,~$(a.b.c)$ אינו שייך ל~$𝓢(Γ)$ משום שהוא שלשה סדורה ולא זוג סדור,
ואילו~$(a(b(c)))$ אינו שייך לקבוצה, משום שהוא אינו עונה על הדרישה שבין פריטים
ימצא סימן הנקודה.

ניתן לאפיין את הקבוצה~$𝓢(Γ)$ באמצעות שני כללי היסק:

\begin{definition}[ביטויי~S מעל אלפאבית]
  \תגית|הגדרה:S| בהנתן אלפאבית~$Γ$ אזי,~$𝓢(Γ)$, קבוצת
  \ע|ביטויי ה-S מעל אלפאבית~$Γ$|, היא הקבוצה הנוצרת באמצעות שני בנאים
  \begin{enumerate}
    ✦ הבנאי הנולארי, המגדיר את הביטוים ה\ע|אטומיים|, כלומר, ביטויי~\E|S| אשר
    אינם ניתנים לפירוק לביטויי~\E|S| אחרים. בנאי זה מוגדר באמצעות כלל היסק שיש
    לו הנחה אחת בלבד,~$γ∈Γ$.
    \begin{equation*}
      \infer{γ∈𝓢(Γ)}{γ∈Γ}.
    \end{equation*}
    לפי בנאי זה, כל אות ב-$Γ$ היא ביטוי~\E|S| אטומי. הבנאי נקרא נולארי, משום
    שעל פי בנאי זה ניתן "ליצור" איברים של הקבוצה~$𝓢(Γ)$ ללא "שימוש" באיברים
    אחרים של הקבוצה.
    ✦ הבנאי הבינארי, המתאר את המבנה של ביטויי~\E|S| \ע|מורכבים|:
    \begin{equation*}
      \infer{(s₁.s₂)∈𝓢(Γ)}{s₁∈𝓢(Γ) & s₂∈𝓢(Γ)}.
    \end{equation*}
    לפי בנאי זה, אם~$s₁$ ו-$s₂$ הם ביטויי~\E|S| (שיכולים להיות אטומיים או
    מורכבים) אזי~$(s₁.s₂)$ הוא ביטוי~\E|S| (מורכב). בנאי זה נקרא בנאי
    בינארי, משום שבכלל ההיסק המתאר אותו
    יש שתי הנחות, וכל אחת מתייחסת לאיברים של~$𝓢(Γ)$. במילים אחרות, על פי בנאי
    זה ניתן "ליצור" איבר חדש של הקבוצה~$𝓢(Γ)$ מתוך "שימוש" ב-\ע|שני| איברים
    קיימים של הקבוצה.
  \end{enumerate}
\end{definition}

